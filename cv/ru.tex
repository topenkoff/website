\documentclass[a4paper]{article}
    \usepackage{fullpage}
    \usepackage{enumitem}
    \usepackage{amsmath}
    \usepackage{textcomp}
    \usepackage{hyperref}
    \usepackage[utf8]{inputenc}
    \usepackage[T2A]{fontenc}
    \textheight=10in
    \pagestyle{empty}
    \raggedright
    \usepackage[left=0.8in,right=0.8in,bottom=0.8in,top=0.8in]{geometry}

\newcommand{\lineunder} {
    \vspace*{-8pt} \\
    \hspace*{-18pt} \hrulefill \\
}

\newcommand{\header} [1] {
    {\hspace*{-18pt}\vspace*{6pt} \textsc{#1}}
    \vspace*{-6pt} \lineunder
}

\begin{document}
\vspace*{-40pt}

%================== Profile ==================%
\vspace*{-10pt}
\begin{center}
    \textbf{{\huge \scshape {Денис Кайшев}}}\\
    \vspace*{10pt}
    \begin{tabular}{l l}
        Москва, Россия & \url{https://github.com/topenkoff} \\
        \href{mailto:topenkoff@gmail.com}{topenkoff@gmail.com} & \url{https://t.me/topenkoff} \\
    \end{tabular}
    \vspace*{15pt}
\end{center}

%================== Experience ==================%
\header{Опыт работы}
\vspace{1mm}

%\textbf{TBD} \\
%\text{TBD} \hfill TBD TBD - TBD TBD\\
%\begin{itemize}[leftmargin=16pt,itemsep=0pt,topsep=0pt,label={-}]
%    \item TBD
%    \item TBD
%\end{itemize}
%\begin{table}[h!]
%    \begin{tabular}{ l l }
%        Tech stack:& \\
%        Languages  & TDB \\
%        Frameworks & TDB \\
%        Tools      & TDB \\
%    \end{tabular}
%\end{table}
%\vspace{\baselineskip}

\textbf{Rambler\&Co} \\
\text{Инженер разработчик, Отдел бэкенд разработки Медиа и Сервисы} \hfill Июнь 2021 - Наст. время\\
\vspace{-1mm}
\begin{itemize}[leftmargin=16pt,itemsep=0pt,topsep=2pt,label={-}]
    \item Cовместно с командой, разработал архитектуру и реализовал оркестратор для пайплайна новостного контента
    \item Обновил систему для аналитики - создание нового сервиса поверх clickhouse'a, миграция исторических данных, реализация real-time обновления данных
    \item Обновление формата представления новостей - переход на унифицированный формат хранения и обработки текста новости c валидацией структуры, добавлена поддержка обратной совместимости со старым форматом, обновление парсера новостей; разработал библиотеку для конвертации Draft.js в HTML на Rust
    \item Переход на PostgreSQL 14 - очистка базы данных от неиспользуемых таблиц/связей, исправление использования расширений (ltree, btree\_gist), обновление diesel до 2.0
    \item Обновление core API - полный рефакторинг с поддержкой async-await синтаксиса и обновление версий actix/futures/tokio
\end{itemize}
\text{Младший инженер разработчик, Отдел бэкенд разработки Медиа и Сервисы} \hfill Апрель 2020 - Июнь 2021\\
\begin{itemize}[leftmargin=16pt,itemsep=0pt,topsep=-8pt,label={-}]
    \item Разработка распределенного пайплайна для подготовки новостного контента
    \item Оптимизация CI/CD и ускорение прохождения пайплайнов проектов(\~2.5-3 раза)
    \item Создание CI пайплайна для сборки python package c CFFI для macOS на linux серверах
\end{itemize}
\begin{table}[h!]
    \begin{tabular}{ l l }
        Tech stack:& \\
        Languages  & rust, python, golang \\
        Frameworks & actix, tokio, serde, diesel, asyncio, aiohttp, fastapi \\
        Tools      & postgresql, redis, clickhouse, rabbitmq, docker, k8s, ci/cd \\
    \end{tabular}
\end{table}

\textbf{МГТУ «СТАНКИН»} \\
\text{Лектор, Кафедра информационных технологий и вычислительных систем} \hfill Сентябрь 2021 - Январь 2022\\
\begin{itemize}[leftmargin=16pt,itemsep=0pt,topsep=0pt,label={-}]
    \item Чтение лекций по дисциплине «Web-программирование»
    \item Составление учебного плана, подготовка материалов для лабораторных работ/семинаров
\end{itemize}
\vspace{\baselineskip}

\textbf{Яндекс.Практикум} \\
\text{Code Reviewer, курс Мидл Python-разработчик} \hfill Сентябрь 2020 - Сентябрь 2021\\
\begin{itemize}[leftmargin=16pt,itemsep=0pt,topsep=0pt,label={-}]
    \item Проверка написанного студентом кода и отправка feedback\textquotesingle{}a по написанному коду
    \item Переработка флоу проверки работ, составление чеклистов проверки, для других ревьюверов
    \item Набор ревьюверов в команду, онбординг
\end{itemize}
\vspace{\baselineskip}

\textbf{Ostrovok.ru} \\
\text{Junior Python Developer, Marketing department} \hfill Июль 2019 - Март 2020\\
\begin{itemize}[leftmargin=16pt,itemsep=0pt,topsep=-8pt,label={-}]
    \item Разработка и поддержка сервиса для продвижения SPA в поисковых системах
    \item Разработка сервиса для SEO-саттелитов
    \item Доработка сервисов для интеграции отдельного SEO-сайта с основным продуктом
\end{itemize}
\begin{table}[h!]
    \begin{tabular}{ l l }
        Tech stack:& \\
        Languages  & python \\
        Frameworks & django \\
        Tools      & postgresql, redis, nginx, docker \\
    \end{tabular}
\end{table}

%================== Education ==================%
\header{Образование}
\textbf{МГТУ «СТАНКИН»}\hfill Москва, Россия\\
Магистр, Информатика и вычислительная техника \hfill Сентябрь 2019 - Июнь 2021\\
\vspace{2mm}
\textbf{МГТУ «СТАНКИН»}\hfill Москва, Россия\\
Бакалавр, Информационные системы и технологии \hfill Сентябрь 2015 - Июнь 2019\\
\vspace{2mm}

\newpage

%================== Projects ==================%
\header{Projects}
{\textbf{\href{https://github.com/topenkoff/poetry-release}{poetry-release}}}\\
Release managment plugin for poetry\\

\vspace*{2mm}
{\textbf{\href{https://github.com/hit-box/hitbox}{hitbox}}}\\
A high-performance caching framework\\
\vspace*{2mm}

%{\textbf{\href{https://github.com/ebalancer/ebalancer}{ebalancer}}}\\
%High-performance reverse-proxy server\\

%================== Activities ==================%
\header{Activities}
{\textbf{\href{https://www.youtube.com/watch?v=gpf_KOAmgzY&t=3040s}{RamblerMeetup\&Python}}}\\
Рассказывал про возможность использования Rust вместо С для написания расширений для Python
\vspace*{2mm}

\end{document}
